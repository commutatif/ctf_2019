\documentclass[10pt,sans,usenames,dvipsnames,french,compress]{beamer}

\usepackage[french]{babel}
\usepackage[utf8]{inputenc}
\usepackage[T1]{fontenc}
\usepackage{lmodern}
\usepackage{graphicx}
\usepackage[absolute,overlay]{textpos}
\usepackage{dirtree}
\usepackage{listings}
\usepackage{pythonhighlight}

\mode<presentation>

\usetheme{Warsaw}
\useoutertheme[subsection=false]{miniframes}

% Fix the crop bullets in the headline
\makeatletter
\setbeamertemplate{headline}
{
  \vskip-0.8ex
  \begin{beamercolorbox}{section in head/foot}
  \vskip2pt\insertnavigation{\paperwidth}\vskip4pt
  \end{beamercolorbox}%
}
\makeatother

%infolines footer
\defbeamertemplate*{footline}{infolines theme}
{
	\leavevmode%
	\hbox{%
	\begin{beamercolorbox}[wd=.333333\paperwidth,ht=2.25ex,dp=1ex,center]{author in head/foot}%
	 \usebeamerfont{author in head/foot}\insertshortauthor%~~(\insertshortinstitute)
	\end{beamercolorbox}%
	\begin{beamercolorbox}[wd=.333333\paperwidth,ht=2.25ex,dp=1ex,center]{title in head/foot}%
	 \usebeamerfont{title in head/foot}\insertshorttitle
	\end{beamercolorbox}%
	\begin{beamercolorbox}[wd=.333333\paperwidth,ht=2.25ex,dp=1ex,right]{date in head/foot}%
	 \usebeamerfont{date in head/foot}\insertshortdate{}\hspace*{2em}
	 \insertframenumber{} / \inserttotalframenumber\hspace*{2ex}
	\end{beamercolorbox}}%
	\vskip0pt%
}

% Terminal-like listing formatting
\lstdefinestyle{Term}
{
    backgroundcolor=\color{black},
    basicstyle=\scriptsize\color{white}\ttfamily,
    gobble=22
}

\graphicspath{{images/}}

\setbeamersize{text margin left=15pt,text margin right=15pt}

% Disable navigation icons
\beamertemplatenavigationsymbolsempty

% Titlepage
\title{CTF SeCReTS}
\subtitle{Présentation}
\author{Alexis Brouste -- Loïc Martin -- Alizée Murat}
\date{Mercredi 20 mars 2019, 16:00:00 CET}
\institute[UVSQ]{UVSQ}

% See for colors names: https://en.wikibooks.org/wiki/LaTeX/Colors

\begin{document}

\section{Introduction}
\subsection{Page de titre}
\begin{frame}[plain]
	\titlepage
\end{frame}

\subsection{Défis présentés}
\begin{frame}{Défis présentés}
	\begin{itemize}
		\item 114 - Reversing 3
		\item 118 - WPA forensics
		\item 120 - Courbe elliptique et AES-CCM
	\end{itemize}
\end{frame}

\section{114 - Reversing 3}
\subsection{Titre}
\begin{frame}
	\begin{beamercolorbox}[sep=8pt,center]{title}
		\usebeamerfont{title}{114 - Reversing 3}
	\end{beamercolorbox}
\end{frame}

\section{118 - WPA forensic}
\subsection{Titre}
\begin{frame}
	\begin{beamercolorbox}[sep=8pt,center]{title}
		\usebeamerfont{title}{118 - WPA forensic}
	\end{beamercolorbox}
\end{frame}

\subsection{Découverte}
\begin{frame}
	\begin{block}{Fichiers}
		\begin{small}
			\dirtree{%
				.1 118-wpa-forensics/.
				.2 indice.txt.
				.2 README.txt.
				.2 {\color<3->{OliveGreen}psk-01.cap}.
			}
		\end{small}
	\end{block}

	\begin{block}<2->{\textit{indice.txt}}
		\begin{itshape}	
			\begin{small}
				In the ennemy base, there is a building, which is guarded by hundreds of 
				highly trained soldiers.

				In this building, there is a WPA-PSK protected wifi with an unbreakable
				password, like {\color<3->{Dandelion}a 10 letter english word (all uppercase) followed by 2
				digits}.

				In this wifi network, there is a web server that only the flag keeper
				(=admin) may access.

				{\color<3->{Dandelion}The flag is his password}

				Good luck!

				PS: By the way, a geek visitor just told me he was walking around 
				the base with his linux laptop (of course, his wlan interface was 
				in  monitor-mode by default). Anyways, he gave me this capture:
				{\color<3->{OliveGreen}psk-01.cap}
			\end{small}
		\end{itshape}
	\end{block}
\end{frame}

\begin{frame}{Procédure}
	\begin{itemize}
		\item Déchiffrement du trafic
		\visible<2->{
			\begin{itemize}
				\item Génération de dictionnaire
				\item Cassage par force-brute de la clef WPA
			\end{itemize}
		}
		\item Analyse du trafic
		\visible<3->{
			\begin{itemize}
				\item Récupération du drapeau
			\end{itemize}
		}
	\end{itemize}
\end{frame}

\subsection{Déchiffrement}
\subsubsection{Reconnaissance}
\begin{frame}[fragile]{Reconnaissance}
	\begin{block}{Analyse du fichier pcap}
		\vspace{-3mm}
		\begin{lstlisting}[style=Term]
			~# aircrack-ng psk-01.cap
		\end{lstlisting}
		\vspace{-2mm}
	\end{block}

	\begin{block}{Liste des \textit{handshakes}}<2->
		\vspace{-3mm}
		\begin{lstlisting}[style=Term]
			Opening psk-01.cap please wait...
			Read 5817 packets.

			   #  BSSID              ESSID                     Encryption

			   1  A0:F3:C1:20:A7:B9  flaghall                  WPA (1 handshake)

			Choosing first network as target.

			Opening psk-01.caplease wait...
			Read 5817 packets.

			1 potential targets

			Please specify a dictionary (option -w).
		\end{lstlisting}
		\vspace{-2mm}
	\end{block}
\end{frame}

\subsubsection{Génération de dictionnaire}
\begin{frame}{Recherche de dictionnaire}
	\begin{block}{Recherche de dictionnaire}
		Moteur de recherche : ``\textit{most common english words txt}''
	\end{block}

	\begin{exampleblock}{Dépôt Github contenant des dictionnaires}<2->
		\textit{This repo contains a list of the 10,000 most common English words in order of frequency, as determined by n-gram frequency analysis of the Google's Trillion Word Corpus.}
	\end{exampleblock}

	\begin{block}{Modification du dictionnaire}<3->
		\begin{itemize}
			\item Garder uniquement les mots de dix lettres
			\item Transcrire en majuscules
			\item Concaténer deux chiffres
		\end{itemize}
	\end{block}
\end{frame}

\begin{frame}[fragile]{Modification du dictionnaire}
	\vspace{-1mm}
	\begin{block}{Script Python}
		\begin{python}
			with open("dict.txt", "r") as iff:
				with open("generated_dict.txt", "w") as off:
					for line in iff:
						line = line.strip().upper()
						if len(line) == 10:
							for n in range(100):
								off.write("{}{:02d}\n".format(line, n))
		\end{python}
		\vspace{-2mm}
	\end{block}
	\vspace{-1mm}
	\begin{block}{Dictionnaire généré}<2->
		\vspace{-3mm}
		\begin{lstlisting}[style=Term]
			~$ wc -l generated_dict.txt
			61000 generated_dict.txt
			~$ cat generated_dict.txt
			...
			MANAGEMENT98
			MANAGEMENT99
			TECHNOLOGY00
			TECHNOLOGY01
			TECHNOLOGY02
			...
		\end{lstlisting}
		\vspace{-2mm}
	\end{block}
\end{frame}

\subsubsection{Cassage de la clef WPA}
\begin{frame}[fragile]{Cassage de la clef WPA}
	\begin{block}{Commande}
		\vspace{-3mm}
		\begin{lstlisting}[style=Term]
			~# aircrack-ng -w generated_dict.txt psk-01.cap
		\end{lstlisting}
		\vspace{-2mm}
	\end{block}

	\begin{block}{Cassage en cours}<2->
		\vspace{-3mm}
		\begin{lstlisting}[style=Term]
			                          Aircrack-ng 1.5.2 

		      [00:00:00] 3852/60999 keys tested (5333.67 k/s) 

		      Time left: 10 seconds                                      6.31%

		                       Current passphrase: ASSESSMENT91               


		      Master Key     : 66 A8 42 94 79 45 03 6F C9 A1 11 33 AA 52 55 D0 
		                       3E 26 2F 55 43 A4 EC 36 E3 13 3C D6 17 B0 96 AC 

		      Transient Key  : CE CF A3 E6 2F 92 8C 5E 94 10 3F 8E 10 F9 FC 92 
		                       25 D3 FE F2 7F D1 4F 27 C1 51 71 8D F5 75 17 5C 
		                       08 F4 5C 2C C8 42 F1 BD 20 3B D7 79 03 2F 05 90 
		                       16 E1 E9 92 BB 73 6A 93 6E F8 7A 80 2F 51 81 9E 

		      EAPOL HMAC     : FC 99 11 B1 45 5C E1 B6 74 E5 2D CB A5 88 21 E1
		\end{lstlisting}
		\vspace{-2mm}
	\end{block}
\end{frame}

\begin{frame}[fragile]{Cassage de la clef WPA}
	\vspace{-1mm}
	\begin{block}{Succès}
		\vspace{-3mm}
		\begin{lstlisting}[style=Term]
		                              Aircrack-ng 1.5.2 

		      [00:00:11] 60926/60999 keys tested (5250.19 k/s) 

		      Time left: 0 seconds                                      99.88%

		                         KEY FOUND! [ PROTECTION78 ]


		      Master Key     : 58 55 68 7D CA B8 C1 48 9E 85 0C E6 AD EB C2 74 
		                       4B 58 6E D5 42 56 BD CF 60 36 F4 3D 2E 13 14 BA 

		      Transient Key  : FF 4D 05 EF 35 C9 91 08 27 C0 CE 25 C2 A0 9D 39 
		                       23 46 41 E5 E9 FD B7 38 57 59 52 94 11 5F 54 05 
		                       6F 6C 3F CA 15 19 BD 0C 40 B7 67 10 6E 41 35 A1 
		                       17 A8 4D 41 BE 48 F9 AB 59 B5 1E 51 09 F6 79 7F 

		      EAPOL HMAC     : 9B F5 0B 0A 75 01 EC F2 C5 CC 64 A6 4E FD 11 16
		\end{lstlisting}
		\vspace{-2mm}

	\end{block}
	\vspace{-1mm}
	\begin{exampleblock}{En bref}<2->
		\begin{itemize}
			\item Clef : \textbf{PROTECTION78}
			\item<3-> Importance d'un dictionnaire optimisé (cassage en onze secondes !)
		\end{itemize}
	\end{exampleblock}
\end{frame}


\subsection{Analyse du trafic}


\section{120 - Courbe elliptique et AES-CCM}
\subsection{Titre}
\begin{frame}
	\begin{beamercolorbox}[sep=8pt,center]{title}
		\usebeamerfont{title}{120 - Courbe elliptique et AES-CCM}
	\end{beamercolorbox}
\end{frame}

\end{document}
